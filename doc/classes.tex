%%%%%%%%%%%%%%%%%%%%%%%%%%%%%%%%%%%%%%%%%%%%%%%%%%%%%%%%%%%%%%%%%%%%%%%%%%%%%
%%
%%  construct.tex            CRISP documentation           Burkhard H\"ofling
%%
%%  @(#)$Id$
%%
%%  Copyright (C) 2000, Burkhard H\"ofling, Mathematisches Institut,
%%  Friedrich Schiller-Universit\"at Jena, Germany
%%
%%%%%%%%%%%%%%%%%%%%%%%%%%%%%%%%%%%%%%%%%%%%%%%%%%%%%%%%%%%%%%%%%%%%%%%%%%%%%
\Chapter{Set theoretical classes}

In {\CRISP}, a class (in the set-theoretical sense) is usually represented
by an algorithm which decides membership in that class. Wherever this makes
sense, sets (see "ref:Set" in the {\GAP} reference
manual) may also be used as classes.


%%%%%%%%%%%%%%%%%%%%%%%%%%%%%%%%%%%%%%%%%%%%%%%
\Section{Creating set theoretical classes}

\>IsClass(<C>) C

returns true if <C> is a class object. The category of class objects is a
subcategory of the category `IsListOrCollection'.

\>Class(<rec>) O
\>Class(<func>) O

returns a class <C>. In the first form, <rec> must be a record having at most
two components, `\\in' and `name'. The values of these components, if
present, are bound to the attributes `MemberFunction' and `Name' of the
class created. The value bound to `\\in' must be a function <func> which
returns `true' if a {\GAP} object belongs to <C>, and `false' otherwise. The
second form is equivalent to `Class(rec(\\in := <func>))'. It is the user's
responsibility to ensure that <func> returns the same result for different
{\GAP} objects representing the same mathematical object (or element, in the
{\GAP} sense; see "ref:objects and elements" in the {\GAP} reference manual).

\beginexample
gap> RequirePackage ("crisp");
                                                    
              _____  ___    __  ____   ___        
             / ___/ / _ \  / / / __/  / _ \       
            / /__  /   _/ / / _\_ \  / ___/       
            \___/ /_/\_\ /_/ /____/ /_/           

              A GAP 4 share package for           
          Computing with Radicals, Injectors        
            Schunck classes and Projectors          
               of finite soluble gropus             
                                                    
                By Burkhard H\"ofling               
                                                    
                For help, type ?CRISP               
                                                    
true
gap> FermatPrimes := Class (p -> IsPrime (p) and p = 2^LogInt (p, 2) + 1); 
Class (in=function( p ) ... end)
\endexample

\>`<obj> in <class>'{element test}!{for classes}
\>IsMember(<obj>, <class>) A

\index{in!for classes}%
\index{membership test!for classes}%
returns true or false, depending whether <obj> belongs to <class> or not. If
bound, `MemberFunction' is used to perform this test. If <obj> can store
attributes, the outcome of the membership test is stored in an attribute
`ComputedIsMembers' of <obj>.

\>`<C1> = <C2>'{equality!for classes}
Since it is not possible to compare classes given by membership algorithms,
two classes are equal in {\GAP} if and only if they are the same {\GAP}
object (see "ref:IsIdenticalObj" in the {\GAP} reference manual).

\>`<C1> \<\ <C2>'{comparison!for classes}

The operation `\<' for classes has no mathematical meaning; it only exists
so that one can form sorted lists of classes.


%%%%%%%%%%%%%%%%%%%%%%%%%%%%%%%%%%%%%%%%%%%%%
\Section{Properties of classes}

\>IsEmpty(<C>)!{for classes} P

This property may be set to `true' or `false' if the class <C> is empty
resp. not empty.

\>IsNonEmpty(<C>) P

This property may be set to `true' or `false' if the class <C> is not
empty resp. empty.

\>MemberFunction(<C>) A

This attribute, if bound, stores a function which decides if an object
belongs to <C> or not, and returns `true' and `false' accordingly. This
function, if bound, is called by the default method for `\\in'.

%%%%%%%%%%%%%%%%%%%%%%%%%%%%%%%%%%%%%%%%%%%%%%%
\Section{Lattice operations for classes}

\>Complement(<C>) O

returns the unary complement of the class <C>, that is, the class consisting
of all objects not in <C>. <C> may also be a set.

\beginexample
gap> cmpl := Complement([1,2]);
Complement ([ 1, 2 ])
gap> Complement (cmpl);
[ 1, 2 ]
\endexample

\>Intersection(<list>)!{of classes} F
\>Intersection(<C1>, <C2>, \dots)!{of classes} F

returns the intersection of the groups in <list>, resp. of the classes
<C1>, <C2>, \dots. If one of the classes is a list, then the result will be
a sublist of that list.

\beginexample
gap> Intersection (Class (IsPrimeInt), [1..10]);
[ 2, 3, 5, 7 ]
gap> Intersection (Class (IsPrimeInt), Class (n -> n = 2^LogInt (n+1, 2) - 1));
Intersection ([ Class (in=function( n ) ... end), 
  Class (in=function( n ) ... end) ])
\endexample

\>Union(<C>, <D>) F

returns the union of <C> and <D>.  

\beginexample
gap> Union (Class (n -> n mod 2 = 0), Class (n -> n mod 3 = 0));
Union ([ Class (in=function( n ) ... end), Class (in=function( n ) ... end) ])
\endexample

\>Difference(<C>, <D>) O

returns the difference of <C> and <D>. If <C> is a list, then the
result will be a sublist of <C>.

\beginexample
gap> Difference (Class (IsPrimePowerInt), Class (IsPrimeInt));
Intersection ([ Class (in=function( n ) ... end), 
  Complement (Class (in=function( n ) ... end)) ])
gap> Difference ([1..10], Class (IsPrimeInt));
[ 1, 4, 6, 8, 9, 10 ]
\endexample

%%%%%%%%%%%%%%%%%%%%%%%%%%%%%%%%%%%%%%%%%%%%%%%%%%%%%%%%%%%%%%%%%%%%%%%%%%%%%
%%
%E
%%

